\begin{Solbook}{1}
 \par \noindent  Dựa vào bảng biến thiên ta thấy khoảng nghịch biến của hàm số là $(-2;0)$.  \par \noindent \selectC \hfill \qedEX 
\end{Solbook}
\begin{Solbook}{2}
 \par \noindent  Xét $y=\dfrac {1}{3} x^{3}+\left (m^{2}-m+2\right ) x^{2}+\left (3 m^{2}+1\right ) x$.\\ Tập xác định $\mathscr D=\mathbb {R}$.\\ Ta có $y^{\prime }=x^{2}+2\left (m^{2}-m+2\right ) x+\left (3 m^{2}+1\right )$.\\ Hàm số đạt cực tiểu tại $x=-2$ nên $y'(-2)=0$ và $y''(-2)>0$.\\ Ta có $4-4\left (m^{2}-m+2\right )+3 m^{2}+1=0 \Leftrightarrow m^{2}-4 m+3=0\Leftrightarrow \hoac {&m=1\\&m=3.}$\\ $y^{\prime \prime }=2 x+2\left (m^{2}-m+2\right )$$y^{\prime \prime }(-2)=2 m^{2}-2 m$.\\ $y^{\prime \prime }(-2)>0 \Leftrightarrow 2 m^{2}-2 m>0 \Leftrightarrow \left [\begin {array}{l}m>1 \\ m<0.\end {array}\right .$\\ Vậy giá trị cần tìm là $m=3$.  \par \noindent \selectB \hfill \qedEX 
\end{Solbook}
\begin{Solbook}{3}
 \par \noindent  Ta có $y'= \dfrac {m^2-9}{(x+m)^2}$.Hàm số $y=\dfrac {mx+9}{x+m}$ nghịch biến trên khoảng $(-\infty ; 1)$ khi và chỉ khi \[ \heva {&m^2-9<0\\ &-m\notin (-\infty ; 1)} \Leftrightarrow \heva {&m^2<9\\ &-m\ge 1} \Leftrightarrow \heva {&-3<m<3\\ &m\le -1} \Leftrightarrow -3<m\le -1. \] Do $m$ nguyên nên $m\in \{-2; -1\}$. Vậy có $2$ giá trị nguyên của $m$ thỏa mãn bài toán.  \par \noindent \selectB \hfill \qedEX 
\end{Solbook}
\begin{Solbook}{4}
 \par \noindent $y'=-3x^2+6mx-3(2m-1)$. \\Hàm số nghịch biến trên $\mathbb {R} \Leftrightarrow \begin {cases} a<0 \\ \Delta ' <0 \end {cases} \Leftrightarrow \begin {cases} -1<0 \\ 9m^2-18m+9 <0 \end {cases} \Leftrightarrow 9(m-1)^2<0 \; (\text {vô lí}) $\\ Vậy không có giá trị của $m$ thỏa ycbt. \par \noindent \selectC \hfill \qedEX 
\end{Solbook}
\begin{Solbook}{5}
 \par \noindent  Từ bảng biến thiên ta suy ra hàm số đã cho đồng biến trên khoảng $(-\infty ;-2)$ và $(0;2)$.  \par \noindent \selectD \hfill \qedEX 
\end{Solbook}
\begin{Solbook}{6}
 \par \noindent  Ta có $f'(x)=3x^2-3$ và $f'(x)=0\Leftrightarrow \hoac {&x=1\in [-1;3]\\&x=-1\in [-1;3].}$\\ Do $f(-1)=4$; $f(1)=0$; $f(3)=20$ nên $\max \limits _{[-1;3]} f(x)=f(3)=20$. \par \noindent \selectA \hfill \qedEX 
\end{Solbook}
\begin{Solbook}{7}
 \par \noindent Do $\lim \limits _{x \to \pm \infty } =1$ nên loại đáp án $B, C$.\\ Do $\lim \limits _{x \to x_0^-} = -\infty $ và $\lim \limits _{x \to x_0^+} = +\infty $ nên chọn $A$.  \par \noindent \selectD \hfill \qedEX 
\end{Solbook}
\begin{Solbook}{8}
 \par \noindent  Tập xác định của hàm số là $\mathscr D = \mathbb {R}$. Khi đó $y' = -4x^3 + 4x$.\\ Ta có $y' = 0 \Leftrightarrow -4x^3 + 4x = 0 \Leftrightarrow \hoac {&x=-1\\&x=1\\&x=0.}$\\ Bảng biến thiên của hàm số là \begin {center} \begin {tikzpicture} \tkzTabInit [nocadre=false,lgt=1.2,espcl=2.5,deltacl=0.6] {$x$/0.6,$y’$/0.6,$y$/2} {$-\infty $,$-1$,$0$,$1$,$+\infty $} \tkzTabLine { ,+,$0$,-,$0$,+,$0$,-, } \tkzTabVar {-/$-\infty $,+/$4$,-/$3$,+/$4$,-/$-\infty $} \end {tikzpicture} \end {center} Dựa vào bảng biến thiên thì hàm số đồng biến trên khoảng $(-\infty ; -1)$ và $(0;1)$.  \par \noindent \selectA \hfill \qedEX 
\end{Solbook}
\begin{Solbook}{9}
 \par \noindent  Ta có $ y' = (m + 2)x^2 + 4(m + 1)x + (m - 5) $.\\ Đồ thị $ (C) $ có hai điểm cực trị nằm về hai phía của trục tung $ \Leftrightarrow (m + 2)( m - 5) < 0 \Leftrightarrow - 2 < m < 5 $.\\ Vì $ m \in \mathbb {Z}^+ $ nên $ m \in \left \{ 1; 2; 3; 4 \right \} $.  \par \noindent \selectA \hfill \qedEX 
\end{Solbook}
\begin{Solbook}{10}
 \par \noindent  Đạo hàm: $y'=\dfrac {-1-m}{(x-1)^{2}}$. \\ Với $-1-m>0\Rightarrow m<-1\Rightarrow \min \limits _{[2;4]}y=y(2)\Rightarrow \dfrac {2+m}{1}=3\Rightarrow m=1\Rightarrow $ loại. \\ Với $-1-m<0\Rightarrow m>-1\Rightarrow \min \limits _{[2;4]}y=y(4)\Rightarrow \dfrac {4+m}{3}=3\Rightarrow m=5\Rightarrow m>4$. \par \noindent \selectC \hfill \qedEX 
\end{Solbook}
\begin{Solbook}{11}
 \par \noindent \begin {itemize} \item ý A. đúng \item ý B. sai vì chiều ngược lại sai. \item ý C. sai vì có thể xảy ra trường hợp $f'(x)=0\, ,\forall x \in (a,b)$ khi đó $f(x)$ là hàm hằng. \item ý D. hiển nhiên sai. \end {itemize} \par \noindent \selectB \hfill \qedEX 
\end{Solbook}
\begin{Solbook}{12}
 \par \noindent  Ta có $ y' = \dfrac {x }{ \sqrt {x^2 + 1}} - m $.\\ Hàm số đồng biến trên $ \mathbb {R} \Leftrightarrow m \leq \dfrac {x}{ \sqrt {x^2 + 1} }, \, \forall x \in \mathbb {R}$. \quad \quad $ (1) $\\ Xét $ f(x) = \dfrac {x}{ \sqrt {x^2 + 1}} $, ta có $ f'(x) = \dfrac {1}{ (x^2 + 1) \sqrt {x^2 + 1} } > 0, \forall x \in \mathbb {R} $.\\ Ta có $ \heva {& \lim \limits _{x \to - \infty } \dfrac {x}{ \sqrt {x^2 + 1} } = - 1 \\ & \lim \limits _{x \to + \infty } \dfrac {x}{ \sqrt {x^2 + 1} } = 1 } $. Ta có bảng biến thiên \begin {center} \begin {tikzpicture}[font=\footnotesize , line join=round, line cap=round, >=stealth, scale = 1] \tkzTabInit [nocadre=false,lgt=1.2,espcl=4.5,deltacl=0.6]{$x$ /0.6,$ f'(x) $ /0.6, $ f(x) $/2}{$ -\infty $, $+\infty $} \tkzTabLine {,+,} \tkzTabVar {-/$-1$,+/$1$} \end {tikzpicture} \end {center} Từ bảng biến thiên, theo $ (1) $ ta được $ m \leq -1 $. \par \noindent \selectD \hfill \qedEX 
\end{Solbook}
\begin{Solbook}{13}
 \par \noindent  \begin {itemize} \item Giải nghiệm phương trình bậc 2 ở tử của đạo hàm phương án A: $4x^2+16x+7=0 \Leftrightarrow \left [\begin {array}{l} x=-\dfrac {1}{2}\smallskip \\ x=-\dfrac {7}{2} \end {array}\right .$\\ Phương trình bậc $2$ này có $2$ nghiệm đơn đồng thời không là nghiệm của mẫu nên đều là hai cực trị. Do đó loại đáp án này. \item Phương án B có: $b^2-3ac=3^2-3.1.(-6)=27>0$. Như vậy phương án này có $2$ cực trị. Loại. \item Phương án C có: $y=\dfrac {2x-1}{x}=2-\dfrac {1}{x}$\\ $\Rightarrow y'=\dfrac {1}{x^2}>0, \forall x \in \mathbb {R}\setminus \{0\}$\\ Do đó hàm số này không có cực trị. \item Phương án D có: $-\dfrac {b}{2a}=-\dfrac {-1}{2.(-1)}=-\dfrac {1}{2}<0$ nên có một cực trị đạt được tại $x=0$. \\ thay vào đạo hàm cấp $2$ ta được cực trị này là cực đại: $y''(0)=-2<0$. \end {itemize}  \par \noindent \selectC \hfill \qedEX 
\end{Solbook}
\begin{Solbook}{14}
 \par \noindent  Xét hàm số $y=f(2-x)-(1-m) x-6$ trên khoảng $(2 ; 3)$.\\ Ta có $y'=-f'(2-x)-1+m=-(2-x)^{3}+3(2-x)-1-1+m$ $\Rightarrow y'=x^{3}-6 x^{2}+9 x-4+m$.\\ Để hàm số $y=f(2-x)-(1-m) x-6$ nghịch biến trên khoảng $(2 ; 3)$ thì \[ y' \leq 0, \forall x \in (2 ; 3) \Leftrightarrow m \leqslant -x^{3}+6 x^{2}-9 x+4, \forall x \in (2 ; 3).\] Xét hàm số $g(x)=-x^{3}+6 x^{2}-9 x+4$ trên khoảng $(2 ; 3)$.\\ Ta có $g'(x)=-3 x^{2}+12 x-9$; $g'(x)=0 \Leftrightarrow \left [ \begin {array}{l}x=1 \notin (2 ; 3) \\ x=3 \notin (2 ; 3).\end {array}\right .$ \begin {center} \begin {tikzpicture}[scale=1] \tkzTabInit [nocadre=false,lgt=1.2,espcl=2.5,deltacl=0.6] {$x$ /0.6,$g'(x)$ /0.6,$g(x)$ /2} {$-\infty $, $1$, $2$, $3$,$+\infty $} \tkzTabLine { ,-,0,+,t,+,0,- } \tkzTabVar {+/$+\infty $,-/ $0$,R/t,+/ $4$,-/$-\infty $} \tkzTabIma {2}{4}{3}{$2$} \end {tikzpicture} \end {center} Từ bảng biến thiên suy ra $m \leqslant 2$.\\ Vì $m \in [-5 ; 5] \Rightarrow m \in [-5 ; 2]$. \\ Mà $ m \in \mathbb {Z}$ nên có $8$ giá trị của tham số $m$.  \par \noindent \selectC \hfill \qedEX 
\end{Solbook}
\begin{Solbook}{15}
 \par \noindent  Tập xác định $\mathscr {D}=\mathbb {R}\setminus \{m\}$.\\ Ta có $y'=\dfrac {x^2-2mx+2m^2(m+1)-2m^3-m^2-1}{(x-m)^2}=\dfrac {x^2-2mx+m^2-1}{(x-m)^2}=\dfrac {g(x)}{(x-m)^2}$.\\ Ta có $y'=0\Leftrightarrow g(x)=0\Leftrightarrow \hoac {&x=m-1\\&x=m+1}\Leftrightarrow \hoac {&m=x+1\\&m=x-1}$\\ Đường thẳng đi qua hai điểm cực trị của đồ thị $(C_m)$ có phương trình $\Delta \colon y=2x-2m(m+1)$.\\ Bảng biến thiên \begin {center} \begin {tikzpicture}[scale=1, font=\footnotesize , line join=round, line cap=round, >=stealth] \tkzTabInit [nocadre=false,lgt=1.2,espcl=2.5,deltacl=0.6] {$x$/0.6,$y'$/0.6,$y$/2} {$-\infty $,$m-1$,$m$,$m+1$,$+\infty $} \tkzTabLine {,+,0,-,d,-,0,+,} \tkzTabVar {-/$-\infty $,+/$-2m^2-2$,-D+/$-\infty $/$+\infty $,-/$-2m^2+2$,+/$+\infty $} \end {tikzpicture} \end {center} Điểm cực đại là $B(m-1;-2m^2-2)$, điểm cực tiểu là $C(m+1;-2m^2+2)$.\\ Quỹ tích điểm cực đại là $\heva {&x=m-1\\&y=-2m^2-2}\Leftrightarrow \heva {&m=x+1\\&y=-2(x+1)^2-2}\Rightarrow y=-2x^2-4x-4 \quad (P_1)$.\\ Quỹ tích điểm cực tiểu là $\heva {&x=m+1\\&y=-2m^2+2}\Leftrightarrow \heva {&m=x-1\\&y=-2(x-1)^2+2}\Rightarrow y=-2x^2+4x\quad (P_2)$.\\ Điểm $A$ vừa là điểm cực đại vừa là điểm cực tiểu nên $A=(P_1)\cap (P_2)$.\\ Phương trình hoành độ giao điểm của $(P_1)$ và $(P_2)$ là $$-2x^2-4x-4=-2x^2+4x\Leftrightarrow x=-\dfrac {1}{2}\Rightarrow y=-\dfrac {5}{2}\Rightarrow A\left (-\dfrac {1}{2};-\dfrac {5}{2}\right ).$$ Điểm cố định của đường $(d)\colon x-(a+1)y+a=0$ là $H(1;1)$ $\Rightarrow \overrightarrow {AH}=\left (\dfrac {3}{2};\dfrac {7}{2}\right )$.\\ Khoảng cách từ $A$ đến đường thẳng $(d)$ lớn nhất khi và chỉ khi $$AH\perp (d)\Leftrightarrow \dfrac {3}{2}\cdot (a+1)+\dfrac {7}{2}\cdot 1=0\Leftrightarrow a=-\dfrac {10}{3}.$$  \par \noindent \selectD \hfill \qedEX 
\end{Solbook}
\begin{Solbook}{16}
 \par \noindent  Đạo hàm của hàm số này là một tam thức bậc $2$ nên nếu nó có hai nghiệm thì đó là hai nghiệm đơn do đó cũng sẽ là hai cực trị của hàm số ban đầu. \\ $x_A^2+x_B^2=2 \Leftrightarrow (x_A+x_B)^2-2x_A.x_B=2 \\ \Leftrightarrow S^2-2P=2$\\ với $S=x_A+x_B, \, P=x_A.x_B$.\\ $y'=x^2-2mx-1=0 \Rightarrow \begin {cases} \Delta '=m^2-(-1).1=m^2+1>0 ,\, \forall m \in \mathbb {R}\\ S=2m \\ P=-1 \end {cases}$\\ Thay vào phương trình trên ta có: \\ $(2m)^2-2.(-1)=2 \Leftrightarrow 4m^2=0 \Leftrightarrow m=0$  \par \noindent \selectA \hfill \qedEX 
\end{Solbook}
\begin{Solbook}{17}
 \par \noindent  Xét đa thức bậc bốn: $g(x)=2f(x)\cdot f''(x)-\left [f'(x)\right ]^2$. \\ Ta có: $g'(x)=2f(x)\cdot f'''(x)=12f(x)$\\ Vì $f(x)=0$ có ba nghiệm phân biệt nên $g'(x)=0$ cũng có ba nghiệm phân biệt. \\ Mặt khác, $g'(x)$ là hàm bậc ba (vì $g(x)$ là hàm bậc bốn) nên ta suy ra $g'(x)=0$ có ba nghiệm bậc nhất. Hay $g(x)$ có tối đa ba cực trị. \\ Điều này dẫn đến $g(x)$ có tối da bốn nghiệm. \\ Gọi $x_1<x_2<x_3$ là ba nghiệm phân biệt của $f(x)$ hay của $g'(x)$. \\ Vì các nghiệm này đều là nghiệm bậc nhất nên hàm $y=f(x)$ không thể đạt cực trị tại các vị trí này (tức hai phía của $x_0 \in \{x_1;x_2;x_3\}$ không thể cùng dương hoặc cùng âm), từ đó ta suy ra hàm số $y=f(x)$ không thể đạt cực trị tại các điểm này, mà $y=f(x)$ là hàm đa thức nên: $$\begin {cases} f'(x_1) \ne 0 \\ f'(x_2) \ne 0 \\ f'(x_3) \ne 0 \end {cases}$$ Ta có bảng biến thiên sau: \begin {center} \begin {tikzpicture} \tkzTabInit [lgt=2,espcl=2.5,nocadre=false] {$x$/0.8,$g'(x)$/0.8, $g(x)$/2} {$-\infty $,$x_1$,$x_2$,$x_3$,$+\infty $} \tkzTabLine {,-,0,+,0,-,0,+,} \tkzTabVar {+/ $+\infty $ ,-/ $g(x_1)$ , +/$g(x_2)$, -/ $g(x_3)$, +/ $+\infty $} \end {tikzpicture} \end {center} Vì: $\begin {cases} g(x_1)=2f(x_1)\cdot f''(x_1)-\left [f'(x_1)\right ]^2=-\left [f'(x_1)\right ]^2 <0\\ g(x_2)=2f(x_2)\cdot f''(x_2)-\left [f'(x_2)\right ]^2=-\left [f'(x_2)\right ]^2 <0\\ g(x_3)=2f(x_3)\cdot f''(x_3)-\left [f'(x_3)\right ]^2=-\left [f'(x_3)\right ]^2 <0 \end {cases}$\\ Suy ra $g(x)=0$ có nhiều nhất $2$ nghiệm phân biệt. \\ Vậy phương trình đầu bài cũng sẽ có nhiều nhất $2$ nghiệm phân biệt.  \par \noindent \selectB \hfill \qedEX 
\end{Solbook}
\begin{Solbook}{18}
 \par \noindent  Xét hàm số $h(x)=f(x^2)-2x$ trên $\mathbb {R}$.\\ Ta có $h'(x)=2x\cdot f(x^2)-2$.\\ Vì $x^2\ge 0,\forall x\in \mathbb {R}$ nên từ đồ thị hàm số $y=f'(x)$ ta có $f'(x^2) \ge 0,\forall x\in \mathbb {R}$.\\ Với $x\le 0$ ta luôn có $h'(x)=2x\cdot f(x^2)-2<0$.\\ Với $x>0$, ta có $h'(x)=0\Leftrightarrow f'(x^2)=\dfrac {1}{x}$.\qquad $(1)$\\ Đặt $t=x^2$, với $t>0$, phương trình $(1)$ trở thành $f'(t)-\dfrac {1}{\sqrt {t}}=0$.\qquad $(2)$\\ Khi $t> 1$ thì $f'(t)\ge 1$ và $\dfrac {1}{\sqrt {t}}<1$ nên phương trình $(2)$ không có nghiệm $t>1$.\\ Xét $t\in (0;1]$, hàm số $k(t)=f'(t)-\dfrac {1}{\sqrt {t}}$ liên tục.\\ Ta có $k'(t)=f''(t)+\dfrac {1}{2\sqrt {t^3}}>0$ với $t\in (0;1)$ nên hàm số $k(t)$ đống biến trên $(0;1)$.\\ Mặt khác $\lim \limits _{x\to 0^+}k(t)=-\infty $ và $k(1)=2$, nên phương trình $(2)$ có nghiệm duy nhất $t=t_0\in (0;1)$.\\ Khi đó $h'(x)=0\Leftrightarrow x=\sqrt {t_0}$.\\ Mặt khác $h(0)=f(0)-0=0$ và $h(2)=f(4)-4>0$ nên ta có bảng biến thiên của hàm số $y=h(x)$ \begin {center} \begin {tikzpicture} \tkzTabInit [nocadre=false,lgt=1.2,espcl=2.5,deltacl=0.6] {$x$ /0.6, $h'$ /0.6, $h(x)$ /2} {$-\infty $,$0$,$\sqrt {t_0}$,$2$,$+\infty $} \tkzTabLine {,-,t,-,$0$,+,t,+,} \path (N12) node[below](A){$+\infty $} ($(N22)!0.5!(N23)$) node(B){$0$} (N33) node[above](C){$h\left (\sqrt {t_0}\right )$} ($(N42)!0.5!(N43)$) node(D){$h(2)$} (N52) node[below](E){$+\infty $}; \draw [-stealth] (A)--(B); \draw [-stealth] (B)--(C); \draw [-stealth] (C)--(D); \draw [-stealth] (D)--(E); \draw [dotted] (N22)--(B); \draw [dotted] (N42)--(D); \end {tikzpicture} \end {center} Từ bảng biến thiên ta có hàm số $y=h(x)$ có một điểm cực trị và đồ thị hàm số $y=h(x)$ cắt $Ox$ tại hai điểm phân biệt, suy ra hàm số $y=g(x)=\left |h(x)\right |$ có ba điểm cực trị, trong đó có hai điểm cực tiểu.  \par \noindent \selectB \hfill \qedEX 
\end{Solbook}
\begin{Solbook}{19}
 \par \noindent  Đặt $ h( x )=f( {x^3} )-3x$.\\ $h'( x )=3{x^2}f'( {x^3} )-3x$.\\ $h'( x )=0\Leftrightarrow f'( {x^3} )=\dfrac {1}{{x^2}}\Leftrightarrow f'( x )=\dfrac {1}{\sqrt [3]{{x^2}}}$.\\ Xét sự tương giao của đồ thị hàm $ y=f'( x )$ và hàm $ y=\dfrac {1}{\sqrt [3]{{x^2}}}$. \begin {center} \begin {tikzpicture}[>=stealth,line join=round,line cap=round,font=\footnotesize ,scale=.5] \draw [->] (-6,0) -- (0,0)node[below left] {$O$} --(6,0)node[below]{$x$}; \draw [->] (0,-6) -- (0,6)node[right]{$y$}; \draw [name path = c2,smooth,samples=100] plot[domain=-4:.4] (\x ,{1/4*(29/6*(\x )^3+29*(\x )^2+87/2*(\x ))-1}); \draw [name path = c1, smooth,samples=100] plot[domain=.1:5] (\x ,{1/(\x )^(2/3)}); \draw [smooth,samples=100] plot[domain=-5:-.1] (\x ,{1/((\x )^2)^(1/3)}); \draw [name intersections={of=c1 and c2,by=x}] [dashed] (x)--($(x)+(-90:2.7)$) node [below] {\footnotesize $a$}; \fill (x) circle (1.5pt); \end {tikzpicture} \end {center} Dựa vào đồ thị ta có $h'( x )=0\Leftrightarrow x=a>0$.\\ Bảng biến thiên hàm số $ h( x )$ là \begin {center} \begin {tikzpicture} \tkzTabInit [nocadre,lgt=1.5,espcl=3,deltacl=0.5] {$x$/.6, $h'(x)$/.6, $h(x)$/3} {$-\infty $,$0$,$a$,$+\infty $} \tkzTabLine {,-,,-,0,+,} \node at (2,-1.5) (a) {}; \node at (8,-4) (b) {$h(a)$}; \node at (11,-1.5) (c) {}; \node at (8,-1.5) (d) {$-h(a)$}; \node at (2,-2.75) (g) {}; \node at (11,-2.75) (h) {}; \node at (5,-3) {$0$}; \path (intersection of g--h and a--b) coordinate (e) (intersection of g--h and c--b) coordinate (f) ; \draw [red, very thick] (g)--(h); \draw [->,>={Stealth[round]},shorten >=1pt] (a)--(b); \draw [->,>={Stealth[round]},shorten >=1pt] (b)--(c); \draw [->,>={Stealth[round]},shorten >=1pt] (e)--(d); \draw [->,>={Stealth[round]},shorten >=1pt] (d)--(f); \end {tikzpicture} \end {center} Dựa vào bảng biến thiên hàm $h(x)$ ta thấy hàm $g(x)=|h(x)|$ có 3 điểm cực trị.  \par \noindent \selectA \hfill \qedEX 
\end{Solbook}
\begin{Solbook}{20}
 \par \noindent  Ta có $f'(x)=2(x-1)(x+m^2)+(x-1)^2\ge 0, \forall x\in [1;2]$ nên $f(x)$ đồng biến trên đoạn $[1;2]$.\\ Ta lại có $f(1)=-\dfrac {3}{2}m$, $f(2)=m^2-\dfrac {3}{2}m+2$ nên ta xét các tường hợp sau \begin {itemize} \item Nếu $m\le 0$ thì $f(x)\ge 0, \forall x\in [1;2]$ nên $\min \limits _{x \in [1;2]} |f(x)|=-\dfrac {3}{2}m$ và $\max \limits _{x \in [1;2]} |f(x)|=m^2-\dfrac {3}{2}m+2$.\\ Khi đó yêu cầu bài toán trở thành $$m^2-\dfrac {3}{2}m+2-\dfrac {3}{2}m=\dfrac {9}{4}\Leftrightarrow m^2-3m-\dfrac {1}{4}=0\Leftrightarrow \hoac {& m=\dfrac {3-\sqrt {10}}{2} \\ & m=\dfrac {3+\sqrt {10}}{2} \ \text {(loại).}}$$ \item Nếu $1\le m\le 2$ thì $\min \limits _{x \in [1;2]} |f(x)|=0$ và $\max \limits _{x \in [1;2]} |f(x)|=\dfrac {3}{2}m$. \\ Yêu cầu bài toán trở thành $0+\dfrac {3}{2}m=\dfrac {9}{4}\Leftrightarrow m=\dfrac {3}{2}$. \item Nếu $0<m<1$ hoặc $m>2$ thì $\min \limits _{x \in [1;2]} |f(x)|=0$ và $\max \limits _{x \in [1;2]} |f(x)|=m^2-\dfrac {3}{2}m+2$.\\ Yêu cầu bài toán trở thành $0+m^2-\dfrac {3}{2}m+2=\dfrac {9}{4}\Leftrightarrow m=\dfrac {3\pm \sqrt {13}}{4} \ \text {(loại)}$. \end {itemize} Vậy $S=\dfrac {3}{2}+\dfrac {3-\sqrt {10}}{2}=\dfrac {1}{2}\left ( \sqrt {36}-\sqrt {10}\right ) $ nên $a=36$, $b=10$ suy ra $\dfrac {a}{b}=\dfrac {18}{5}$.  \par \noindent \selectD \hfill \qedEX 
\end{Solbook}
\begin{Solbook}{21}
 \par \noindent  Ta có $g(x)=\left [f(x)\right ]^2+\left [f'(x)\right ]^2-2f(x) \cdot f'(x)=\left [f(x)-f'(x)\right ]^2$.\\ Đa thức $f(x)=ax^4+bx^3+cx^2+\mathrm {\,d}x+e$ tiếp xúc với trục hoành tại hai điểm $x=-1$, $x=2$.\\ Do đó $f(x)=a(x+1)^2(x-2)^2$ với $a>0$.\\ Do $\lim \limits _{x\to +\infty } f(x)=+\infty $, suy ra $f'(x)=a\left [2(x+1)(x-2)^2+2(x-2)(x+1)^2\right ]$. {\allowdisplaybreaks \begin {eqnarray*} f(x)-f'(x)&=&a(x+1)^2(x-2)^2-a \left [2(x+1)(x-2)^2+2(x-2)(x+1)^2\right ]\\ &=& a(x+1)(x-2)\left [(x+1)(x-2)-2(x-2)-2(x+1)\right ]\\ &=& a(x+1)(x-2)\left (x^2-5x\right )\\ &=& a(x+1)(x-2)x(x-5). \end {eqnarray*} } Do đó $g(x)=\left [f(x)-f'(x)\right ]^2=a^2(x+1)^2x^2(x-2)^2(x-5)^2$ có $7$ điểm cực trị. \par \noindent \selectC \hfill \qedEX 
\end{Solbook}
\begin{Solbook}{22}
 \par \noindent  Xét hàm số $y=f(x)=x^3-mx+1, f'(x)=3x^2-m$. \\ Đồ thị hàm số $y=\left |f(x)\right |=\left |x^3-mx+1\right |$ được dựng từ đồ thị của hàm số $y=f(x)$ bằng cách giữ lại phần đồ thị hàm số trên trục $Ox$ và lấy đối xứng phần phía dưới $Ox$ qua trục $Ox$, xóa bỏ phần đồ thị nằm phía dưới trục $Ox$ của $y=f(x)$. \begin {itemize} \item Với $m=0$ ta có hàm số: $y=x^3+1 \Rightarrow \begin {cases} y'=3x^2 \geq 0\\ f(1)=1^3+1=2>0 \end {cases} \\ \Rightarrow $ Hàm số $y=\left |f(x)\right |=\left |x^3-mx+1\right |$ đồng biến trên $\left [1;+\infty \right )$. \\ Vậy $m=0$ thỏa yêu cầu. \item Với $m>0$ ta có: $f'(x)=0$ có $2$ nghiệm phân biệt $x_1;x_2 (x_1 < x_2)$. \\ Bảng biến thiên: \begin {center} \begin {tikzpicture} \tkzTabInit [lgt=2,espcl=3.5,nocadre=false] {$x$/0.8,$f'(x)$/0.8, $f(x)$/2} {$-\infty $,$x_1$,$x_2$,$+\infty $} \tkzTabLine {,+,0,-,0,+,} \tkzTabVar {-/ $-\infty $ , +/ $f(x_1)$, -/ $f(x_2)$ , +/ $+\infty $} \end {tikzpicture} \end {center} Để hàm số $y=\left |x^3-mx+1\right |$ đồng biến trên $\left [1;+\infty \right )$ thì: $$\begin {cases} m>0 \\ x_1<x_2 \leq 1 \\ f(1) \geq 0 \end {cases} \Leftrightarrow \begin {cases} m>0 \\ -\dfrac {m}{3}+1 \geq 0 \\ 2-m \geq 0 \end {cases} \Leftrightarrow 0<m \leq 2$$ Mà $m \in \mathbb {N} \Rightarrow m \in \{1;2\}$\\ Vậy: $S=\{ 0;1;2\}$. Số phần tử của $S$ là $3$. \end {itemize}  \par \noindent \selectA \hfill \qedEX 
\end{Solbook}
\begin{Solbook}{23}
 \par \noindent  Gọi $l_1,l_2$ lần lượt là chu vi tam giác đều và chu vi hình vuông.\\ Gọi $d_1$ là độ dài cạnh hình vuông.\\ Gọi $S_1,S_2,S$ lần lượt là diện tích tam giác đều, diện tích hình vuông, tổng diện tích của hai hình thu được.\\ Chu vi tam giác đều trên là: $l_1=3a$\\ Chu vi hình vuông là: $l_2=6-3a$\\ Cạnh hình vuông là:$d_1=\dfrac {l_2}{4}=\dfrac {6-3a}{4}$\\ Diện tích tam giác là: $S_1=\dfrac {a^2\sqrt {3}}{4}$\\ Diện tích hình vuông là: $S_2=d_1^2=\left (\dfrac {6-3a}{4}\right )^2$ \\ Tổng diện tích các hình thu được là: $S=\dfrac {a^2\sqrt {3}}{4}+\left (\dfrac {6-3a}{4}\right )^2$\\ $S'(a)=\dfrac {a \sqrt 3}{2}+\dfrac {1}{16}.2.(-3)(6-3a)=\dfrac {a\sqrt 3}{2}-\dfrac {3}{8}(6-3a)=a\left ( \dfrac {\sqrt 3}{2}+\dfrac {9}{8}\right )-\dfrac {9}{4}$\\ $S'(a)=0 \Leftrightarrow a=\dfrac {54-24 \sqrt 3}{11}=\dfrac {18}{9+4\sqrt {3}}$  \par \noindent \selectC \hfill \qedEX 
\end{Solbook}
\begin{Solbook}{24}
 \par \noindent  Vận tốc của vật được tính bởi: $v(t)=-t^2+12t$.\\ Ta có $v'(t)=-2t+12$.\\ Bảng biến thiên: \begin {center} \begin {tikzpicture} \tkzTabInit [nocadre=false,lgt=1,espcl=3] {$t$ /0.8, $v'$ /.8,$v$ /1.8} {$0$ ,$6$ ,$9$} \tkzTabLine {,+,0,-,} \tkzTabVar { -/$0$,+/$36$,-/$27$} \end {tikzpicture} \end {center} Dựa vào bảng biến thiên ta có vận tốc lớn nhất của vật đạt được bằng $36$ m/s.  \par \noindent \selectB \hfill \qedEX 
\end{Solbook}
\begin{Solbook}{25}
 \par \noindent  $G'(x)=2.0,025x-1= 0 \Leftrightarrow x=20\\ G''(x)=2.0,025>0 \Rightarrow G''(20)=0,05$\\ Như vậy $G(x)$ đạt cực tiểu thì $x=20 \,$mg.  \par \noindent \selectA \hfill \qedEX 
\end{Solbook}
